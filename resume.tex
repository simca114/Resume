% Alexander Simchuk's Resume
% Created: 9 Mar, 2014

\documentclass[11pt,oneside]{article}
\usepackage{geometry}
\usepackage[T1]{fontenc}

\pagestyle{empty}
\geometry{letterpaper,tmargin=1in,bmargin=1in,lmargin=1in,rmargin=1in,headheight=0in,headsep=0in,footskip=.3in}

\setlength{\parindent}{0in}
\setlength{\parskip}{0in}
\setlength{\itemsep}{0in}
\setlength{\topsep}{0in}
\setlength{\tabcolsep}{0in}

% Name and contact information
\newcommand{\name}{Alexander Simchuk}
\newcommand{\addr}{7206 SW Florence LN, Portland OR 97223}
\newcommand{\phone}{(503) 928-1829}
\newcommand{\email}{sim6@pdx.edu}

%%%%%%%%%%%%%%%%%%%%%%%%%%%%%%%%%%%%%%%%%%%%%%%%%%%%%%%%%
% New commands and environments

% This defines how the name looks
\newcommand{\bigname}[1]{
	\begin{center}\fontfamily{phv}\selectfont\Huge\scshape#1\end{center}
}

% A ressection is a main section (<H1>Section</H1>)
\newenvironment{ressection}[1]{
	\vspace{4pt}
	{\fontfamily{phv}\selectfont\Large#1}
	\begin{itemize}
	\vspace{3pt}
}{
	\end{itemize}
}

% A resitem is a simple list element in a ressection (first level)
\newcommand{\resitem}[1]{
	\vspace{-4pt}
	\item \begin{flushleft} #1 \end{flushleft}
}

% A ressubitem is a simple list element in anything but a ressection (second level)
\newcommand{\ressubitem}[1]{
	\vspace{-1pt}
	\item \begin{flushleft} #1 \end{flushleft}
}

% A resbigitem is a complex list element for stuff like jobs and education:
%  Arg 1: Name of company or university
%  Arg 2: Location
%  Arg 3: Title and/or date range
\newcommand{\resbigitem}[3]{
	\vspace{-5pt}
	\item
	\textbf{#1}---#2 \\
	\textit{#3}
}

% This is a list that comes with a resbigitem
\newenvironment{ressubsec}[3]{
	\resbigitem{#1}{#2}{#3}
	\vspace{-2pt}
	\begin{itemize}
}{
	\end{itemize}
}

% This is a simple sublist
\newenvironment{reslist}[1]{
	\resitem{\textbf{#1}}
	\vspace{-5pt}
	\begin{itemize}
}{
	\end{itemize}
}


%%%%%%%%%%%%%%%%%%%%%%%%%%%%%%%%%%%%%%%%%%%%%%%%%%%%%%%%%
% Now for the actual document:

\begin{document}
\fontfamily{ppl} \selectfont
% Name with horizontal rule
\bigname{\name}
\vspace{-8pt} \rule{\textwidth}{1pt}
\vspace{-1pt} {\small\itshape \addr \hfill \phone; \email}
\vspace{8 pt}


%%%%%%%%%%%%%%%%%%%%%%%%
\begin{ressection}{Experience}
    \begin{ressubsec}{Computer Action Team}{Portland State University}{Systems Administor: September 2012 - present}
		\ressubitem{Manage Linux systems for the College of Engineering and Computer Science}
		\ressubitem{Use Puppet and Git extensively}
		\ressubitem{Troubleshoot user issues and document resolution}
	\end{ressubsec}
	\begin{ressubsec}{College of Computer Sciences}{Portland State University}{Computer Science Volunteer Tutor: Spring 2013 - Fall 2013}
        \ressubitem{Assist students with intro level (CS161-CS202) computer science assignments involving Python, C++, and Java}
        \ressubitem{Teach essential programming tools, including: vim, gcc, g++, gdb}
	\end{ressubsec}
\end{ressection}


%%%%%%%%%%%%%%%%%%%%%%%%
\begin{ressection}{Education}
    \begin{ressubsec}{Portland State University}{Portland, OR}{B.S. in Computer Science}
		\ressubitem{Expected graduation date: June 2016}
	\end{ressubsec}
\end{ressection}


%%%%%%%%%%%%%%%%%%%%%%%%
\begin{ressection}{Achievements and Activities}
    \resitem{Experience in working in both Windows and Linux environments, including Debian/Ubuntu, Red Hat and Solaris}
    \resitem{Experience in coding in: C, C++, Java, Python, SQL, bash, HTML, CSS, Javascript}
    \resitem{Experience in setting up services such as: Git, Puppet, NFS, ZFS, MySQL/PostgreSQL}
    \resitem{Experience in writing dynamic HTML forms with Javascript and PHP}
    \resitem{Configuring OpenLDAP server with backend and frontend data replicated from existing infrastructure for a test web application interface}
    \resitem{Updating documantation for setting up an OpenLDAP server in the College of Engeneering and Computer Science environment}
    \resitem{Mentoring lab sessions for setting up basic Unix services including: LAMP stack, Databases, Git, NFS, Nagios}
    \resitem{Wrote C program that uses nCurses to create a GUI for backups retrieval}
\end{ressection}


\end{document}
